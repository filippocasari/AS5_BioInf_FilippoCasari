\documentclass{article}
\usepackage[utf8]{inputenc}
\usepackage{float}
\usepackage{breqn}
\usepackage{graphicx}
\usepackage{subcaption}
\title{Assignment 5 BioInformatics}
\author{Filippo Casari}
\date{November 2022}

\begin{document}

\maketitle

\section{Point 1}
\subsection*{Question}
Terminology review: differentiate DNA, gene, and genome. Your explanation
must demonstrate that you clearly understand what each term means distinctively
of the others. (max. 150 words).
\subsection*{Answer}

\section{Point 2}
Briefly describe gene prediction and why we do it. (max. 150 words)
\subsection*{Answer}
\section{Point 3}
Describe and differentiate the three main classifications of gene prediction
programs: ab initio based, homology based, and consensus based. (max. 150
words)
\subsection*{Answer}

\section{point 4}
In gene prediction, discuss the characteristics in a DNA sequence that have been
statistically shown to point to or act as a kind of marker or clue signifying a coding
region, i.e., genes/exons (max. 150 words).
\subsection*{Answer}
%\begin{figure}[H]
%\includegraphics[width=8cm, angle=90]{b.jpeg}
%\centering
%\end{figure}

\section{Point 5}
Why do you think that gene prediction is a difficult task? (max. 300 words)
\subsection*{Answer}


\end{document}

